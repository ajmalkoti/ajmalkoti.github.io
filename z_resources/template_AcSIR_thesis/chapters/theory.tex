\chapter{Theory}

\section{Background}
When you are preparing the theory part various elements are used for that. A few are discussed in following text. 


\subsection{Equations}
	A simple equation can be written as 
	\begin{align}
		a^2 = b^2 +c ^2     \label{eq:theory_pythagorus}
	\end{align}
	%
	To write above equation the code is 
	\begin{verbatim}
	\begin{align}
	a^2 = b^2 +c ^2     \label{eq:theory_pythagorus}
	\end{align}
	\end{verbatim}
	%	
	This equation can be referred anywhere in the text as following \cref{eq:theory_pythagorus} for which the command is  \verb|\cref{eq:theory_pythagorus}|.


	Very complex equations can also be written using latex easily. 
	Following is an example
	
	\begin{align}
		\nabla \times F = \int_{-\infty}^{t} \lambda K(x) \pdv{s}{x} dx
	\end{align}
	


\subsection{Figures}
	Let us show an example of figure insertion here
	\begin{figure}[!h]
		\center
		\includegraphics[width=.32\textwidth]{theory_vectorize_operator.png}
		\caption{The figure is taken from  \cite{malkoti_algorithm_2018} 
			to show how to include a figure. 
			A uniquie lable  is attached to this figure so that it can be 
			referred anywhere in the thesis.}
		\label{fig:theory_vectorize}
	\end{figure}

	The code used for inserting the figure is 
	\begin{verbatim}	
	\begin{figure}[!h]
		\center
		\includegraphics[width=.32\textwidth]{theory_vectorize_operator.png}
		\caption{The figure is taken from  \cite{malkoti_algorithm_2018} to show how 
		to include a figure. A uniquie lable  is attached to this figure so that it 
		can be referred anywhere in the thesis.}
		\label{fig:theory_vectorize}
	\end{figure}
	\end{verbatim}
	
	The figure can be cited anywhere in the text as \cref{fig:theory_vectorize} using the command \verb|\cref{fig:theory_vectorize}|.



\subsection{Tables}
	Now we demonstrate a table which is showing a list of different ways to cite a figure and table.

\begin{table}[!h]
\caption{ List of new commands provided by this package}
\label{tab:list_ref_cmd}
\center
\begin{tabular}{cc}
	\hline
	\verb|\cref{fig:theory_vectorize} |		&	\cref{fig:theory_vectorize}\\
	\verb|\Cref{fig:theory_vectorize}|		&	\Cref{fig:theory_vectorize}\\
	\verb|\cref{tab:citation_style} |		&	\cref{tab:citation_style}\\
	\verb|\Cref{tab:citation_style}|		&	\Cref{tab:citation_style}\\
	\hline
\end{tabular}
\end{table} 


The above table was created using following code
\begin{verbatim}
	\begin{table}[!h]
	\caption{ List of new commands provided by this package}
	\label{tab:list_ref_cmd}
	\center
	\begin{tabular}{cc}
	\hline
	\verb|\cref{fig:theory_vectorize} |		&	\cref{fig:theory_vectorize}\\
	\verb|\Cref{fig:theory_vectorize}|		&	\Cref{fig:theory_vectorize}\\
	\verb|\cref{tab:citation_style} |		&	\cref{tab:citation_style}\\
	\verb|\Cref{tab:citation_style}|		&	\Cref{tab:citation_style}\\
	\hline
	\end{tabular}
	\end{table} 
\end{verbatim}



\subsection{Units}
The units can be correctly formatted as following\\
\begin{tabular}{cc}
	\hline
	                     Command                      &                   Effect                 \\
	                    \hline
	       \verb|1500 \si{\meter\per\second}|         &        1500 \si{\meter\per\second}        \\
	\verb|$ \SI{3.456}{\Newton \per \meter\squared}$| & $ \SI{3.456}{\newton \per \meter\squared}$\\
	\hline
\end{tabular} 

%\section{Some Random Text Here After}
%\lipsum[1-20]

  