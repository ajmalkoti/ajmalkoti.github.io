\chapter*{Summary}
\vspace*{-3em}


Ideally here should come the summary of the work. I take the liberty of explaining the structure of this package 
\begin{description}
\item[chapters]: This directory contains all the chapters, appendix, etc. 
	You may like to name the chapters as \verb|ch1_Intro|, \verb|ch2_Theory|,..., etc.\\ \textbf{Adding more chapters }. \\
	Step 1: Create a tex file (e.g FileName.tex) in the directory "chapters" \\
	Step 2: Add here command \verb|\include{chapters/FileName}|. 

\item[figs]: This directory contains all the figures. 
           You may like to name the figures with the chapter name before them, e.g.   
           \verb|intro_anomaly_map.jpg|, or  \verb|theory_gravity_response.jpg|.  
           The rule is not necessary, but useful to sort figures when there are many figures, e.g., $>100$.
\item[references]: 
	This directory contains all the references in bibtex format in the file \verb|references/mybibfile.bib|.\\
	Note :  		\\
	1) Make sure there is no duplicate entry in the bibliography file.\\
    2) Building all reference requires special processing sequence which is available in your GUI editor's preferences. Make sure you select-\\
  			Pdflatex---$>$ biblatex ---$>$ Pdflatex ---$>$ Pdflatex ---$>$ ViewPdf




\item[Other] : Elements viz. mathematical equations, figure/table insertion and their referencing, units, etc. are explained in the subsequent chapters. 

\end{description}

