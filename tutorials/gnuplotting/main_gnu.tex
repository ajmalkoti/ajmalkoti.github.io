% Copyright 2004 by Till Tantau <tantau@users.sourceforge.net>.
%
% In principle, this file can be redistributed and/or modified under
% the terms of the GNU Public License, version 2.
%
% However, this file is supposed to be a template to be modified
% for your own needs. For this reason, if you use this file as a
% template and not specifically distribute it as part of a another
% package/program, I grant the extra permission to freely copy and
% modify this file as you see fit and even to delete this copyright
% notice. 

\documentclass{beamer}

% There are many different themes available for Beamer. A comprehensive
% list with examples is given here:
% http://deic.uab.es/~iblanes/beamer_gallery/index_by_theme.html
% You can uncomment the themes below if you would like to use a different
% one:
%\usetheme{AnnArbor}
%\usetheme{Antibes}
%\usetheme{Bergen}
%\usetheme{Berkeley}
%\usetheme{Berlin}    % good one
%\usetheme{Boadilla}  % no header and title but footer
%\usetheme{boxes}
%\usetheme{CambridgeUS}
%\usetheme{Copenhagen}
%\usetheme{Darmstadt}
\usetheme{Dresden}      % navigation shown horizontal in top
%\usetheme{default}
%\usetheme{Frankfurt}
%\usetheme{Goettingen}
%\usetheme{Hannover}
%\usetheme{Ilmenau}
%\usetheme{JuanLesPins}
%\usetheme{Luebeck}
%\usetheme{Madrid}                   %using this one
%\usetheme{Malmoe}
%\usetheme{Marburg}
%\usetheme{Montpellier}
%\usetheme{PaloAlto}
%\usetheme{Pittsburgh}
%\usetheme{Rochester}
%\usetheme{Singapore}	%good
%\usetheme{Szeged}
%\usetheme{Warsaw}

\usepackage{subcaption}
\usepackage{natbib}
\usepackage[percent]{overpic}
\usepackage{color}
\newcommand{\cred}[1]{\textcolor{red}{\textbf{#1}}}
\newcommand{\cblue}[1]{\textcolor{blue}{\textbf{#1}}} 
\newcommand{\cgreen}[1]{\textcolor{green}{\textbf{#1}}} 
\usepackage{amsmath}
\usepackage{hyperref}
\usepackage{multicol}
\usepackage{fancyvrb}  % it respects the tab spacing as well

\title[GNUplot]{\textbf{How to use the GNU plot}}

% A subtitle is optional and this may be deleted
%\subtitle{Optional Subtitle}

\author[Ajay Malkoti]{
		\vspace*{.5cm}
		{\large \textbf{Ajay Malkoti}}\\
		{\footnotesize Senior Research Fellow \\ %10PP13J28006, Jan 2013\\		
				Academy of Scientific and Innovative Research
				 }\\[2em]
		}		 
		%Under the supervison of \\[1em]
		%{\small \textbf{Dr. Nimisha Vedanti} \hspace{3.3cm} \textbf{Dr. R. K. Tiwari}}\\
		%{\footnotesize (Principle Scientist) \hspace{3.3cm}   (Raja Rammanna Fellow)}
		%}
% - Give the names in the same order as the appear in the paper.
% - Use the \inst{?} command only if the authors have different
%   affiliation.

\institute[AcSIR] % (optional, but mostly needed)
	{ 	Shallow Seismic R \& D group\\
	  	CSIR-National geophysical Research Institute
	}
% - Use the \inst command only if there are several affiliations.
% - Keep it simple, no one is interested in your street address.

\date[Nov 2018]
% - Either use conference name or its abbreviation.
% - Not really informative to the audience, more for people (including
%   yourself) who are reading the slides online

\subject{Seismic modeling}
% This is only inserted into the PDF information catalog. Can be left
% out. 

% If you have a file called "university-logo-filename.xxx", where xxx
% is a graphic format that can be processed by latex or pdflatex,
% resp., then you can add a logo as follows:
% \pgfdeclareimage[height=0.5cm]{university-logo}{AcSIR_transparent2.png}
% \logo{\pgfuseimage{university-logo}}

\logo{%
\makebox[0.95\paperwidth]{%
    \includegraphics[width=1cm,height=1cm,keepaspectratio]{ngri_logo1.png}%
    \hfill%
    \includegraphics[width=1.5cm,height=1cm,keepaspectratio]{AcSIR_transparent2.png}~%
  }%
}

% Delete this, if you do not want the table of contents to pop up at
% the beginning of each subsection:
\AtBeginSubsection[]
{
  \begin{frame}<beamer>{Outline}
    \tableofcontents[currentsection,currentsubsection]
  \end{frame}
}

%%%%%%%%%%%%%%%%%%%%%%%%%%%%%%%%%%%%%%%%%%%%%%%%%%%%%%%%%%%%%%%%%%%%%%%%
%%%%%%%%%%%%%%%%%%%%%%%%%%%%%%%%%%%%%%%%%%%%%%%%%%%%%%%%%%%%%%%%%%%%%%%%
%%%%%%%%%%%%%%%%%%%%%%%%%%%%%%%%%%%%%%%%%%%%%%%%%%%%%%%%%%%%%%%%%%%%%%%%
%%%%%%%%%%%%%%%%%%%%%%%%%%%%%%%%%%%%%%%%%%%%%%%%%%%%%%%%%%%%%%%%%%%%%%%%



% Let's get started
\begin{document}
	%
	%
	%
	\begin{frame}
	  	\titlepage
	\end{frame}
	%
	%
	%	
	\begin{frame}
		\tableofcontents
	\end{frame}
	%
	%
	%	
	\section{Introduction}
	\begin{frame}
		Types of plotting\\
		\begin{enumerate}
			\item plotting a function in 1D (line plot)
			\item Plotting data from file
			\item plotting multiple 1D data (overlay and multiplot/subplot)
			\item plotting a function in 2D, 3D as surface, contour etc plot
		\end{enumerate}
	
	\end{frame}
		%
	%
	%
	\begin{frame}[fragile]{The first script}
		\scriptsize
		\begin{itemize}
			\item	\textbf{Single line graph of asin(x)*cos(x)}
			\begin{verbatim}
				reset						# a good practice to add it
				set xlabel "X-AXIS"
				set ylabel "Y-AXIS"
				set xrange [0:5]
				set yrange [-2:2]
				a= 3.14
				print a						# disp value of a on screen
				plot a*sin(x)*cos(x)	
			\end{verbatim}
			
			\item	\textbf{Overlay graph of sin(x)*cos(x) and exp(x)		}	
			\begin{verbatim} 
				set xlabel "X-AXIS"
				set ylabel "Y-AXIS"
				set xrange [0:5]
				set yrange [-2:2]
				a= 20
				print a
				unset yrange        # exponential doent fit in earlier yrange
				plot a*sin(x)*cos(x) title 'sincos',\
				     exp(x) title 'exp'	
			\end{verbatim}
		\end{itemize}
	\end{frame}
	%
	%
	%
	%
	\begin{frame}
		\frametitle{Setting up canvas}
		\footnotesize
		\begin{enumerate}
			\item set title "This is my first plot"\hspace{1cm} \# Title
			\item set xlabel "$x$ axis" 	\hspace{1cm} \# x axis label
			\item set ylabel "$y$ axis"		\hspace{1cm} \# y axis lable
			\item set yrange [-50:2000]
			\item set xrange [-10:100]
			\item set xtics  40
			\item set mxitcs 8		\hspace{1cm} \# minor xtics
			\item set key top right	\hspace{1cm} \# legend
		\end{enumerate}
	\end{frame}
	%
	%
	%
	\begin{frame}
		\frametitle{Some other basic commands}
		\footnotesize
		\begin{enumerate}
			\item plot/splot: for simple/surface plot
			\item exit or quit 	
			\item help {$<$topic$>$} 		
			\item load  e.g.   load 'work.gnu'
			\item replot or refresh \hspace{1cm}\#  repeats the last plot/splot command
			\item unset xrange    \hspace{1cm}\# unsetting xrange to its default
			\item reset			 \hspace{1cm}\# unsetting  all graph related setting
		\end{enumerate}
	\end{frame}
	%
	%
	%	
	\begin{frame}
		\frametitle{Setting output}
		\begin{enumerate}
			\item set terminal postscript landscape color enhanced\\
				set output "my-plot.ps" 
			\item set terminal png\\
				set output "my-plot.png" 
		\end{enumerate}
		
	\end{frame}	
	%
	%
	%		
	\begin{frame}
		running the command/file
		\begin{enumerate}
			\item \textbf{gnuplot$>$ save "myplot.plt"} \\
			Used in GNUplot environment for \textit{saving current setting} to file 'myplot.plt'
			\item \textbf{gnuplot$>$ load "savefile.plt"}\\
			Used in GNUplot environment for \textit{loding the setting} to file 'myplot.plt'
			
			\item  \textbf{\$ gnuplot savefile.plt}\\
			Used in shell envorenment for running the GNUplot script.
			
		\end{enumerate}
	\end{frame}	
	%
	%
	%
	\section{Types of Data}	
	\begin{frame}[fragile,allowframebreaks]
		\frametitle{GNU: 1D data}
		\scriptsize  \center 	\hrule 
		\vspace{.3cm}
		\textbf{Type I- single line	}
		\begin{Verbatim}
				# X     Y
				1.0   1.2
				2.0   1.8
				3.0   1.6	
				
		plot 'data.txt' using 1:2 with lines
		\end{Verbatim}
		\vspace{.3cm}
		\hrule 
		\vspace{.3cm}
		\textbf{Type II- Two line(same sampling)}
		\begin{Verbatim}
				# X    Y1    y2
				1.0   1.2	1.5
				2.0   1.8	2.3
				3.0   1.6	2.0	
				
		plot 'data.txt' using 1:2 with lines linestyle 4,\
		'data.txt' using 1:3 with lines linestyle 6
		\end{Verbatim}
		\vspace{2cm}
		\hrule 
		\vspace{.3cm}
		\textbf{Type III- Two lines(diff sampling)}
		\begin{Verbatim}
			# First data block (index 0)
			#X      Y
			1.0   1.2
			2.0   1.8
			3.0   1.6
			
			# Second index block (index 1)
			#X      Y
			1.0   1.2
			2.0   1.8
			3.0   1.6
			
	set style line 1  linecolor rgb '#0060ad' \
	    linetype 1 linewidth 2  pointtype 7 pointsize 1.5
	set style line 2  linecolor rgb '#dd181f' \
	    linetype 1 linewidth 2  pointtype 5 pointsize 1.5
	plot 'plotting_data3.dat' index 0 with linespoints linestyle 1,\
		'' index 1 with linespoints linestyle 2
		
		OR
		
	set style line 1  lc rgb '#0060ad' lt 1 lw 2  pt 7 ps 1.5
	set style line 2  lc rgb '#dd181f' lt 1 lw 2  pt 5 ps 1.5
	plot 'plotting_data3.dat' index 0 with linespoints ls 1,\
		'' index 1 with linespoints ls 2
		\end{Verbatim}
		\vspace{.3cm}
		\hrule 
		\vspace{.3cm}
		\textbf{Type IV: plot with error}
		\begin{Verbatim}
			 X     Y     Yerror\\
			1.0   1.2   0.1\\
			2.0   1.8   0.1\\
			3.0   1.6   0.1\\
		\end{Verbatim}
		\vspace{.3cm}
		\hrule 
		\vspace{.3cm}
			\textbf{Type V: 1 line, diff error for x,y}\\
		\begin{verbatim}
			 X     Y     EX    EY\\
			1.0   1.2   0.8   1.5\\
			2.0   1.8   0.3   2.3\\
			3.0   1.6   1.0   2.1\\				
		\end{verbatim}
		\vspace{.3cm}
		\hrule 
		\vspace{.3cm}
		\textbf{Type VI: 2 lines, diff sampling, diff error}\\
		\begin{verbatim}
			1.1   0.8   0.2\\
			2.1   0.3   0.2\\
			3.1   1.0   0.2\\
							
			1.2   1.5   0.3\\
			2.2   2.3   0.3\\
			3.2   3.1   0.3\\
		\end{verbatim}
	\end{frame}




	\section{Plotting the function}


	
	\section{GNU plotting data from file}	
		\begin{frame}
			\frametitle{My first plot}
			\textbf{Start GNU}\\
			\$ gnuplot \\[1em]
			\textbf{Give command}\\
			$gnuplot>$ plot "test.dat" using 1:2 with lines,%\
			"test.dat" using 1:3 with lines,%\
			"test.dat" using 1:4 with lines \\[1em]
			
			\textbf{In multiple lines}\\
	 		$gnuplot>$ plot "test.dat" using 1:2 with lines,$\setminus$ \\
			"test.dat" using 1:3 with lines,$\setminus$ \\
			"test.dat" using 1:4 with lines \\[1em]
		\end{frame}	
		%
		%
		%
		%			
		\begin{frame}\frametitle{GNU plotting data from file: 1D data}
			\url{http://lowrank.net/gnuplot/datafile2-e.html}
			%
			\begin{minipage}{.48\textwidth}
				\footnotesize
				\textbf{Data file, data.txt}\\
				$gnuplot>$ 
	
				plot "data.txt" using 1:2 with lines,%\
				
				"data.txt" using 1:3 with lines,%\
			
				"test.txt" using 1:4 with lines \\[1em]
				
				Plotting with new function
				
			\end{minipage}
			%
			\begin{minipage}{.48\textwidth}
				\tiny
				\center
				   X           Y1         Y2         Y3\\
				-1.0000    0.0000     0.0000     1.0000\\
				-0.9000    0.5700     1.1769     0.7150\\
				-0.8000    1.0800     1.4400     0.4600\\
				-0.7000    1.5300     1.4997     0.2350\\
				-0.6000    1.9200     1.4400     0.0400\\
				-0.5000    2.2500     1.2990    -0.1250\\
				-0.4000    2.5200     1.0998    -0.2600\\
				-0.3000    2.7300     0.8585    -0.3650\\
				-0.2000    2.8800     0.5879    -0.4400\\
				-0.1000    2.9700     0.2985    -0.4850\\
				0.0000    3.0000    -0.0000    -0.5000\\
				0.1000    2.9700    -0.2985    -0.4850\\
				0.2000    2.8800    -0.5879    -0.4400\\
				0.3000    2.7300    -0.8585    -0.3650\\
				0.4000    2.5200    -1.0998    -0.2600\\
				0.5000    2.2500    -1.2990    -0.1250\\
				0.6000    1.9200    -1.4400     0.0400\\
				0.7000    1.5300    -1.4997     0.2350\\
				0.8000    1.0800    -1.4400     0.4600\\
				0.9000    0.5700    -1.1769     0.7150\\
				1.0000    0.0000    -0.0000     1.0000
			\end{minipage}	
		\end{frame}	
		%
		%
		%
		\begin{frame}
			\frametitle{Plotting 1D data with error}
			\footnotesize
			gnuplot$>$ plot "test.dat" using 1:2:3 with yerrorbars\\
			gnuplot$>$ plot "test.dat" using 1:2:3:4 with yerrorbars\\[1em]

			\scriptsize
			\begin{tabular}{p{3cm}p{2cm}p{1.1cm}p{3cm}}	
				\textbf{Data Format}	&\textbf{Column}	&\textbf{Using}		&\textbf{With}\\
				(X,Y) data				&X Y		&1:2		&lines, points, steps, linespoints, boxes, etc.\\
				Y $\pm$dY	&X Y dY		&1:2:3		&yerrorbars\\
				X $\pm$ dX	&X Y dX		&1:2:3		&xerrorbars\\
				Y $\pm$ dY, 
				and X $\pm$ dX	&X Y dX dY	&1:2:3:4	&xyerrorbars\\
				Y range [Y1,Y2]	&X Y Y1 Y2	&1:2:3:4	&yerrorbars\\
				X range [X1,X2]	&X Y X1 X2	&1:2:3:4	&xerrorbars\\
				Y range [Y1,Y2], 
				and X range [X1,X2]	&X Y X1 X2 Y1 Y2	& 1:2:3:4:5:6	& xyerrorbars		
			\end{tabular}
		\end{frame}
		%
		%
		%		
		\begin{frame}
			Performing mathematical operation on a coloum and plotting\\
			\begin{itemize}
			\item  plot 'exp.dat' using 1:exp(\$2) with lines \hspace{1cm} or \\
			plot 'exp.dat' u 1:exp(\$2) w lines
			
			\item plot 'exp.dat' using 1:2:(sqrt(\$2)) with yerrorbars \hspace{1cm} or \\
			plot 'exp.dat' u 1:2:(sqrt(\$2)) w yerr
			
			\item f(x) = A0*exp(-x/tau)\\
			A0=1000;tau=1\\
			plot 'exp.dat' u 1:2:(sqrt(\$2)) w yerr, f(x)
			\end{itemize}

			
			
		\end{frame}				
		%
		%
		%
		\begin{frame}
			\frametitle{Setting the line properties}
			\scriptsize
			\begin{enumerate}
				\item \textbf{Viewing styles 4 }\\
					gnuplot$>$show linetype 4\\ 
				    linetype 4,  linecolor rgb "\#e69f00"  linewidth 1.000 dashtype solid pointtype 4 pointsize default pointinterval 0\\
				    
				\item \textbf{Viewing all default styles(1-8)} \\ 
					gnuplot$>$show linetypes\\				    
				
				\item \textbf{Defining new line style}(GNUplot ver. dependent) \\			
					set linestyle  1 lt 1 lc 7 \hspace{1cm}\# black-solid\\
					set linestyle  2 lt 2 lc 1 \hspace{1cm}\# red-dashed
				
			\end{enumerate}
		
		\end{frame}				
		
		%
		%
		%		
		\begin{frame}
		\frametitle{Sample scripts: Multiplot (as subplot)}
			\url{http://gnuplot.sourceforge.net/demo/layout.html}
			\begin{itemize}
			\item	set multiplot; \# get into multiplot mode\\
			 set size 1,0.5;\\
			 set origin 0.0,0.5; plot sin(x);\\
			 set origin 0.0,0.0; plot cos(x)\\
			 unset multiplot 
			\end{itemize}		
		\end{frame}
		%
		%
		%
		\begin{frame}[fragile,allowframebreaks]
		\frametitle{Sample scripts: Multiplot (as subplot)}
			\scriptsize
			\url{http://gnuplot.sourceforge.net/demo/layout.html}
			\begin{itemize}
			
			\item	
			\begin{verbatim}
				set terminal postscript landscape color enhanced
				set output "my-plot.ps" 
				# Set overall margins for the combined set of plots and size them
				# to generate a requested inter-plot spacing
				if (!exists("MP_LEFT"))   MP_LEFT = .1
				if (!exists("MP_RIGHT"))  MP_RIGHT = .95
				if (!exists("MP_BOTTOM")) MP_BOTTOM = .1
				if (!exists("MP_TOP"))    MP_TOP = .9
				if (!exists("MP_GAP"))    MP_GAP = 0.05
				
				set multiplot layout 2,2 columnsfirst title "{/:Bold=15 Multiplot with explicit page margins}" $\setminus$
				              margins screen MP_LEFT, MP_RIGHT, MP_BOTTOM, MP_TOP spacing screen MP_GAP
				
				set format y "\%.1f"
				set key box opaque
				set ylabel 'y'
				set xlabel 'x'
				set xrange [-2*pi:2*pi]
				plot sin(x) lt 1
				plot cos(x) lt 2
				
				unset ylabel
				unset ytics	
				unset xlabel
				plot sin(2*x) lt 3
				
				set xlabel 'x'
				plot cos(2*x) lt 4
				unset multiplot	
			\end{verbatim}
			
					
					
					
			\end{itemize}	
		\end{frame}
		%
		%
		%
	\section{Advanced commands}
		\begin{frame}[fragile,allowframebreaks]{Commands}
			\scriptsize
			(\url{http://soc.if.usp.br/manual/gnuplot-doc/htmldocs/})
			\begin{enumerate}
				\item \textbf{cd} e.g. cd \verb|"c:\newdata"|.   Note: In windows use double quots("..")\\
				\item \textbf{pwd}	
				\item \textbf{save}: It saves user-defined functions, variables, the 
				`set term` status, all `set' options, or all of these, plus the last 
				`plot' (`splot') command to the specified file. 
					\begin{verbatim}
						save 'work.gnu'
						save functions 'func.dat'
						save var 'var.dat'
						save set 'options.dat'
						save term 'myterm.gnu'
						save '-'
						save '|grep title >t.gp'     
					\end{verbatim}
				\item \textbf{set} e.g. \verb|set <option>|\\
				angles,arrow,autoscale,bars,bind \_, bmargin,border,boxwidth,
				clabel,clip,cntrparam,color\_box,colornames,contour,data\_style,
				datafile,decimalsign,dgrid3d,dummy,encoding,fit \_, fontpath,format \_,
				function\_style,functions\_,grid,hidden3d,historysize,isosamples,key,
				label,linetype,lmargin,loadpath,locale,logscale,macros,mapping,margin,mouse,
				multiplot,mx2tics,mxtics,my2tics,mytics,mztics,object,offsets,origin,output,
				parametric\_,plot\_,pm3d,palette,pointintervalbox,pointsize,polar\_,print\_,psdir,
				raxis,rmargin,rrange,rtics,samples,size,style,surface,,table,terminal,
				termoption,tics,ticslevel,ticscale,timestamp,timefmt,title\_,tmargin,trange,urange
				,variables,version,view,vrange,x2data,x2dtics,x2label,x2mtics,x2range,x2tics
				,x2zeroaxis,xdata,xdtics,xlabel,xmtics,xrange,xtics,xyplane,xzeroaxis,y2data
				,y2dtics,y2label,y2mtics,y2range,y2tics,y2zeroaxis,ydata,ydtics,ylabel,ymtics
				,yrange,ytics,yzeroaxis,zdata,zdtics,zzeroaxis,cbdata,cbdtics,zero,zeroaxis,
				,zlabel,zmtics,zrange,ztics,cblabel,cbmtics,cbrange,cbtics\\[5em]


				\item 	\textbf{if}
					\begin{verbatim}
						if (<condition>) { <command>; <command>
				                  <commands>
				           } else {
				                  <commands>
				           }
					\end{verbatim}


				\item \textbf{do iterator}: \\
					\begin{verbatim}
					set multiplot layout 2,2
					do for [name in "A B C D"] {
					    filename = name . ".dat"
					    set title sprintf("Condition \%s",name)
					    plot filename title name }					
					 unset multiplot
					\end{verbatim}
				
					
				\item 	\textbf{for}
					\begin{verbatim}
						for [intvar = start:end{:increment}]
						for [stringvar in "A B C D"]

						plot for [filename in "A.dat B.dat C.dat"] filename using 1:2 with lines
						plot for [basename in "A B C"] basename.".dat" using 1:2 with lines
						set for [i = 1:10] style line i lc rgb "blue"
						unset for [tag = 100:200] label tag
			           
						set for [i=1:9] for [j=1:9] label i*10+j sprintf("%d",i*10+j) at i,j
					\end{verbatim}		
					

					
				\item \textbf{evaluate}: This is especially useful for a repetition of similar commands.\\
					\begin{verbatim}
						set_label(x, y, text) \
						= sprintf("set label '%s' at %f, %f point pt 5", text, x, y)
						eval set_label(1., 1., 'one/one')
						eval set_label(2., 1., 'two/one')
						eval set_label(1., 2., 'one/two')
					\end{verbatim}
         			\vspace{2cm}
         			        			
         			
				\item 	\textbf{fit}
					\begin{verbatim}
						fit {<ranges>} <expression> '<datafile>' {datafile-modifiers}
			               via '<parameter file>' | <var1>{,<var2>,...}
			               
						FIT_LIMIT = 1e-6
						f(x) = a*x**2 + b*x + c
						fit f(x) 'measured.dat' via 'start.par'
						fit f(x) 'measured.dat' using 3:($7-5) via 'start.par'
						fit f(x) './data/trash.dat' using 1:2:3 via a, b, c

						g(x,y) = a*x**2 + b*y**2 + c*x*y						
						fit g(x,y) 'surface.dat' using 1:2:3:(1) via a, b, c
						
						fit a0 + a1*x/(1 + a2*x/(1 + a3*x)) 'measured.dat' via a0,a1,a2,a3
						fit a*x + b*y 'surface.dat' using 1:2:3:(1) via a,b
						fit [*:*][yaks=*:*] a*x+b*yaks 'surface.dat' u 1:2:3:(1) via a,b
						fit a*x + b*y + c*t 'foo.dat' using 1:2:3:4:(1) via a,b,c
						
						h(x,y,t,u,v) = a*x + b*y + c*t + d*u + e*v
						fit h(x,y,t,u,v) 'foo.dat' using 1:2:3:4:5:6:(1) via a,b,c,d,e
					\end{verbatim}
	
			\item 	\textbf{stat (Statistical Summary)} 
					\verb|stats 'filename' [using N[:M]] [name 'prefix'] [[no]output]]|
					It also produces three set of variables
					\begin{verbatim}
						first set: It reports how the data is laid out in the file:
						STATS_records           # total number of in-range data records
						STATS_outofrange        # number of records filtered out by range limits
						STATS_invalid           # number of invalid/incomplete/missing records
						STATS_blank             # number of blank lines in the file
						STATS_blocks            # number of indexable data blocks in the file
						
						
						The second set reports properties of the in-range data from a single column. 
						If the corresponding axis is autoscaled (x-axis for the 1st column, y-axis 
						for the optional second column) then no range limits are applied. 
						If two columns are being analysed in a single `stats` command, the the suffix 
						"_x" or "_y" is appended to each variable name. I.e. STATS_min_x is the minimum 
						value found in the first column, while STATS_min_y is the minimum value found 
						in the second column.
						STATS_min               # minimum value of in-range data points
						STATS_max               # maximum value of in-range data points
						STATS_index_min         # index i for which data[i] == STATS_min
						STATS_index_max         # index i for which data[i] == STATS_max
						STATS_lo_quartile       # value of the lower (1st) quartile boundary
						STATS_median            # median value
						STATS_up_quartile       # value of the upper (3rd) quartile boundary
						STATS_mean              # mean value of in-range data points
						STATS_stddev            # standard deviation of the in-range data points
						STATS_sum               # sum
						STATS_sumsq             # sum of squares
										     
						The third set of variables is only relevant to analysis of two data columns.
										
						STATS_correlation       # correlation coefficient between x and y values
						STATS_slope             # A corresponding to a linear fit y = Ax + B
						STATS_intercept         # B corresponding to a linear fit y = Ax + B
						STATS_sumxy             # sum of x*y
						STATS_pos_min_y         # x coordinate of a point with minimum y value
						STATS_pos_max_y         # x coordinate of a point with maximum y value
										     
						It may be convenient to track the statistics from more than one file at the same time. The `name` option causes the default prefix "STATS" to be replaced by a user-specified string. For example, the mean value of column 2 data from two different files could be compared by
										
						stats "file1.dat" using 2 name "A"
						stats "file2.dat" using 2 name "B"
						if (A_mean < B_mean) {...}
					\end{verbatim}
								
			\end{enumerate}				
		\end{frame}
		%
		%
		%
		\begin{frame}[fragile,allowframebreaks]
			\scriptsize
			\frametitle{Examples}
			\url{http://gnuplot-surprising.blogspot.com/2012/05/how-to-pick-out-maximum-and-minimum.html}
			
			\begin{minipage}{.25\textwidth}
				\textbf{Data}
				\begin{verbatim}
					0.1   0.28901
					0.2   0.05063
					0.3   0.72124
					0.4   0.28427
					0.5   0.50505
					0.6   0.10181
					0.7   0.00846
					0.8   0.36249
					0.9   0.48757
					1.0   0.59509
					1.1   0.86525
					1.2   0.69662
					1.3   0.50589
					1.4   0.33813
					1.5   0.10803
				\end{verbatim}
			\end{minipage}
			%
			%			
			\begin{minipage}{.48\textwidth}
				\begin{verbatim}
					reset
					set term png
					set output "max_min.png"
					stats "data.dat" u 1:2 nooutput
					set xrange [STATS_min_x:STATS_max_x]
					set label 1 "Maximun" at STATS_pos_max_y,STATS_max_y\ 
					                                       offset 1,-0.5
					set label 2 "Minimun" at STATS_pos_min_y,STATS_min_y \
					                                        offset 1,0.5
					plot "data.dat" w p pt 3 lc rgb"#ff0000" notitle, \
					         STATS_min_y w l lc rgb"#00ffff" notitle, \
 					         STATS_max_y w l lc rgb"#00ffff" notitle
					set output
				\end{verbatim}
			\end{minipage}
			\vspace{2cm}
			
			
			\url{http://www.usm.uni-muenchen.de/people/puls/lessons/intro_general/gnuplot/gnuplot_for_beginners.pdf}
			\begin{verbatim}
				f1(x) = a1*tanh(x/b1) 		# define the function to be fit
				 a1 = 300; b1 = 0.005; 		# initial guess for a1 and b1
				 fit f1(x) 'force.dat' using 1:2 via a1, b1
				Final set of parameters Asymptotic Standard Error
				======================= ==========================
				a1 = 308.687 +/- 10.62 (3.442%)
				b1 = 0.00226668 +/- 0.0002619 (11.55%)
				
				
				f2(x) = a2 * tanh(x/b2) # define the function to be fit
				a2 = 300; b2 = 0.005; # initial guess for a and b
				fit f2(x) 'force.dat' using 1:3 via a2, b2
				Final set of parameters Asymptotic Standard Error
				======================= ==========================
				a2 = 259.891 +/- 12.82 (4.933%)
				b2 = 0.00415497 +/- 0.0004297 (10.34%) 


				set key at 0.018,150 title "F(x) = A tanh (x/B)"       # title to key!
				set title "Force Deflection Data \n and curve fit"     # note newline!
				set pointsize 1.5 # larger point!
				set xlabel 'Deflection, {/Symbol D}_x (m)'             # Greek symbols!
				set ylabel 'Force, {/Times-Italic F}_A, (kN)'          # italics!
				plot "force.dat" using 1:2 title 'Column data' with points pt 3, \
				       "force.dat" using 1:3 title 'Beam data' with points pt 4, \
				       a1 * tanh( x / b1 ) title 'Column-fit: A=309, B=0.00227', \
				       a2 * tanh( x / b2 ) title 'Beam-fit: A=260, B=0.00415'
			\end{verbatim}
		\end{frame}

%%				 f2(x) = a2 * tanh(x/b2) # define the function to be fit
%%				 a2 = 300; b2 = 0.005; # initial guess for a and b
%%				 fit f2(x) 'force.dat' using 1:3 via a2, b2
%%				Final set of parameters Asymptotic Standard Error
%%				======================= ==========================
%%				a2 = 259.891 +/- 12.82 (4.933%)
%%				b2 = 0.00415497 +/- 0.0004297 (10.34%) 
%%
%%				set key at 0.018,150 title "F(x) = A tanh (x/B)" # title to key!
%%				set title "Force Deflection Data \n and curve fit" # note newline!
%%				set pointsize 1.5 # larger point!
%%				set xlabel 'Deflection, {/Symbol D}_x (m)' # Greek symbols!
%%				set ylabel 'Force, {/Times-Italic F}_A, (kN)' # italics!
%%				plot "force.dat" using 1:2 title 'Column data' with points pt 3, \
%%				 "force.dat" using 1:3 title 'Beam data' with points pt 4, \
%%				 a1 * tanh( x / b1 ) title 'Column-fit: A=309, B=0.00227', \
%%				 a2 * tanh( x / b2 ) title 'Beam-fit: A=260, B=0.00415'
%		\end{frame}				
%		
		%
		%
		%
		\begin{frame}
			\scriptsize
			\begin{enumerate}
				\item \url{http://lowrank.net/gnuplot/intro/basic-e.html}
				\item \url{http://www.usm.uni-muenchen.de/people/puls/lessons/intro_general/gnuplot/gnuplot_for_beginners.pdf}	
				\item \url{http://gnuplot.sourceforge.net/demo/finance.html}
			\end{enumerate}				
		\end{frame}
\end{document}


